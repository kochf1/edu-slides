%%
%% Copyright 2024 Fernando Koch
%% main.tex, a simple 'Getting Starter'.
%
% This work may be distributed and/or modified under the
% conditions of the LaTeX Project Public License, either version 1.3
% of this license or (at your option) any later version.
% The latest version of this license is in
%   https://www.latex-project.org/lppl.txt
% and version 1.3c or later is part of all distributions of LaTeX
% version 2008 or later.
%
% This work has the LPPL maintenance status `maintained'.
% The Current Maintainer of this work is Fernando Koch.
% This work consists of the file edu-slides.sty and associted testing files.
%
% Documentation and Getting Started:
%   https://github.com/kochf1/edu-slides
%
% Disclaimer: 
% GPT4o has been applied for the development of source code and documentation through this package
%

\documentclass[aspectratio=169]{beamer}
\usepackage{edu-slides}

%% 
%%
%% (Step 1) Adjust the Configurations
%%

\pgfkeys{
    /params/.is family,
    /params/.cd,
        % Author
        author/.initial         = Dr. Fernando Koch,
        author/email/.initial   = kochf@fau.edu,
        author/url/.initial     = http://linkedin.com/in/fkoch,
    % 
        % Course
        course/.initial         = Principles of Software Engineering,
        course/code/.initial    = CEN~4010,
        course/logo/.initial    = logos/cen4010-logo.png,
    %
        % Texts
        texts/locations/.initial   = {., ./images}, % Folder locations to look for Texts
    %
        % Images
        images/locations/.initial = {., ./images, ./logos}, % Folder locations to look for Image
    %   
        % Debug
        debug/layout/.initial   = false,    %% mark positions, so you can adjust the layouts
        debug/compTime/.initial = true,     %% add marker with time of compilation on Cover and BackCover
}

%%
%% (Step 2) Create the sequence of slides and content.
%% You can find more Slide Templates through the Documentation on GitHub.
%%

\begin{document}

    %% Open the Slide Deck
    \slideCover{Lesson 1 \\ Introductions}

    %% Explore Standard templates
    \slideAgenda{Chapter 1 \\ Chapter 2 \\ Chapter 3}
    \slideTeleprompt{slideTeleprompt}{!text-1}
    \slideDouble{slideDouble}{!text-1}{!text-1}
    \slideTextImage{slideTextImage}{img-1}{!text-1}
    \slideImageText{slideImageText}{img-1}{!text-1}
    \slideImage{slideImage}{img-1}
    \slideDoubleImage{slideDoubleImage}{img-1}{img-1}
    \slideBigImage{img-1}

    %% Explore Quote templates
    \slideQuote{Some Other Templates to Explore}
    \slideQuoteText{The centrality of interactions}{!text-1}
    \slideTextQuote{The centrality of interactions}{!text-1}
    \slideQA{Could this really work?}{A: It all depends}
    \slideLiterature{book-cover}{http://somewhere.com/book}{Title}{!book-chapters}

    %% Explore Concept templates
    % Concepts with Text & Explanation
    \slideConceptOne[title=slideConcept, font=big, one=72\%]{!text-1}
    \slideConceptTwo[title=slideConceptTwo, font=big, color=true, one=72\%, two=25\%]{!text-1}{!text-1}
    \slideConceptThree[title=slideConceptThree, font=larger, one=72\%, two=25\%, three=19.5\%]{Explanation 1}{Explanation 2}{Explanation 3}
    \slideConceptFour[title=slideConceptFour, font=large, one=72, two=25, three=19.5, four=10]{Explanation 1}{Explanation 2}{Explanation 3}{Explanation 4}
    
    % Concepts with Image & Explanation
    \slideConceptOne[title=slideConcept (image), type=image, one=fau-logo]{!text-1}
    \slideConceptTwo[title=slideConceptTwo (image), type=image, one=fau-logo, two=fau-logo]{!text-1}{!text-1}
    \slideConceptThree[title=slideConceptThree (image), type=image, one=fau-logo, two=fau-logo, three=fau-logo]{Explanation 1}{Explanation 2}{Explanation 3}
    \slideConceptFour[title=slideConceptFour (image), type=image, one=fau-logo, two=fau-logo, three=fau-logo, four=fau-logo]{Explanation 1}{Explanation 2}{Explanation 3}{Explanation 4}

    %% Close the Slide Deck
    \slideBackCover
\end{document}




