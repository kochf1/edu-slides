%%
%% Copyright 2024 Fernando Koch
%% main.tex, a simple 'Getting Starter'.
%
% This work may be distributed and/or modified under the
% conditions of the LaTeX Project Public License, either version 1.3
% of this license or (at your option) any later version.
% The latest version of this license is in
%   https://www.latex-project.org/lppl.txt
% and version 1.3c or later is part of all distributions of LaTeX
% version 2008 or later.
%
% This work has the LPPL maintenance status `maintained'.
% Documentation and Getting Started:
%   https://github.com/kochf1/edu-slides
%
% Disclaimer: 
% Diverse Generative AI models have been applied during the development of 
% this application.
%

\documentclass[aspectratio=169]{beamer}
\usepackage{edu-slides}

%% 
%%
%% (Step 1) Adjust the Configurations
%%

\pgfkeys{
    /params/.is family,
    /params/.cd,
        % Author
        author/.initial         = Dr. Fernando Koch,
        author/email/.initial   = kochf@fau.edu,
        author/url/.initial     = http://linkedin.com/in/fkoch,
    % 
        % Course
        course/.initial         = Principles of Software Engineering,
        course/code/.initial    = CEN~4010,
        course/logo/.initial    = logos/cen4010-logo.png,
    %   
        % Debug
        debug/import/.initial   = false,     %% detail process of finding and importing files
        debug/layout/.initial   = false,     %% mark positions, so you can adjust the layouts
        debug/time/.initial     = true,      %% add marker with time of compilation on Cover and BackCover
}

%%
%% (Step 2) Create the sequence of slides and content.
%% You can find more Slide Templates through the Documentation on GitHub.
%%

\begin{document}

    %% Open the Slide Deck
    \slideCover{Lesson 1 \\ Introductions}

    \slideImageText[title=Fernando Koch]{fkoch}{
        \begin{itemize}
            \item About me
            \item About This Course
            \item How are we going to work?
            \item What you need to configure?
            \item Survey
        \end{itemize}
    }

    %%    
    %% STANDARD templates

    \slideAgenda{Chapter 1 \\ Chapter 2 \\ Chapter 3}
    \slideTeleprompt[type=diagram, title=slideTeleprompt]{!function-1}
    \slideDouble[title=slideDouble]{!text-1}{!text-1}
    \slideDouble[type=diagram, title=slideDouble]{!tree-1}{!text-1}

    \slideTextImage[type=diagram, title=slideTextImage]{img-1}{!tree-1}
    \slideImageText[type=diagram, title=slideImageText]{img-1}{!tree-1}
    \slideImage[title=slideImage]{img-1}
    \slideDoubleImage[title=slideDoubleImage]{img-1}{img-1}
    \slideBigImage{img-1}

    %%
    %% QUOTE templates
    \slideQuote{Some Other Templates to Explore}
    \slideQuoteText{The centrality of interactions}{!text-1}
    \slideQuoteText[type=diagram]{The centrality of interactions}{!tree-1}
    \slideTextQuote[type=diagram]{The centrality of interactions}{!tree-1}
    \slideQA{Could this really work?}{answer=A: It all depends}
    \slideLiterature[image=book-cover, link=http://somewhere.com/book, title=Title]{!book-chapters}

    %%
    %% DEFINITION Templates    

    \slideDefinitions[type=text,
                  arrow=true,
                  title=Title,
                  title/1=Title,
                  title/2=Title 25,
                  title/3=Title 35,
                  text/1=!text-1,
                  text/2=!text-1]

    \slideProgress[type=text,
                  arrow=true,
                  title=Title,
                  title/1=Title,
                  title/2=Title 25,
                  title/3=Title 35,
                  text/1=!text-1,
                  text/2=!tree-1]

    \slideEvolution[type=text,
                  font=h3,
                  title=Title,
                  title/1=Title,
                  title/2=Title 25,
                  title/3=Title 35,
                  title/4=Title 45,
                 % title/5=Title 55,
                  text/1=Step 1,
                  text/2=Step 2,
                  text/3=Step 3,
                  text/4=Step 4]

    % Close the Slide Deck
    \slideBackCover
\end{document}




