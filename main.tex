%%
%% Copyright 2024 Fernando Koch
%% main.tex, a simple 'Getting Starter' for edu-style.sty
%
% This work may be distributed and/or modified under the
% conditions of the LaTeX Project Public License, either version 1.3
% of this license or (at your option) any later version.
% The latest version of this license is in
%   https://www.latex-project.org/lppl.txt
% and version 1.3c or later is part of all distributions of LaTeX
% version 2008 or later.
%
% This work has the LPPL maintenance status `maintained'.
% The Current Maintainer of this work is Fernando Koch.
% This work consists of the file edu-slides.sty and associted testing files.
%
% Documentation and Getting Started:
%   https://github.com/kochf1/edu-slides
%
% Disclaimer: 
% GPT4o has been applied for the development of source code and documentation through this package
%

\documentclass[aspectratio=169]{beamer}
\usepackage{edu-slides}

%%  

%%
%% (Step 1) Adjust the Configurations
%%

\pgfkeys{
    /params/.is family,
    /params/.cd,
        % Logo
        logo/.initial           = images/fau-logo.png,
    %
        % Banner
        banner/.initial         = images/fau-banner.png,
    %            
        % Author
        author/.initial         = Dr. Fernando Koch,
        author/email/.initial   = kochf@fau.edu,
        author/url/.initial     = http://linkedin.com/in/fkoch,
    % 
        % Course
        course/.initial         = Principles of Software Engineering,
        course/code/.initial    = CEN~4010,
        course/logo/.initial    = images/course-logo.png,
    %   
        % Debug
        debug/layout/.initial   = false,    %% mark positions, so you can adjust the layouts
        debug/compTime/.initial = true,     %% add marker with time of compilation on Cover and BackCover
}

%%
%% (Step 2) Create the sequence of slides and content.
%% You can find more Slide Templates through the Documentation on GitHub.
%%

\begin{document}
    %% Open the Slide Deck
    \slideCover{Lesson 1 \\ Introductions}

    %% Add a few slides
    %\slideAgenda{Chapter 1 \\ Chapter 2 \\ Chapter 3}
    %\slideTeleprompt{slideTeleprompt}{For example:
\begin{itemize}
    \item \textbf{First Point}: A brief description of the first point.
    \item \textbf{Second Point}: A brief description of the second point.
    \item \textbf{Third Point}: A brief description of the third point.
\end{itemize}}
    %\slideDouble{slideDouble}{For example:
\begin{itemize}
    \item \textbf{First Point}: A brief description of the first point.
    \item \textbf{Second Point}: A brief description of the second point.
    \item \textbf{Third Point}: A brief description of the third point.
\end{itemize}}{For example:
\begin{itemize}
    \item \textbf{First Point}: A brief description of the first point.
    \item \textbf{Second Point}: A brief description of the second point.
    \item \textbf{Third Point}: A brief description of the third point.
\end{itemize}}
    %\slideTextImage{slideTextImage}{tests/img-1.png}{For example:
\begin{itemize}
    \item \textbf{First Point}: A brief description of the first point.
    \item \textbf{Second Point}: A brief description of the second point.
    \item \textbf{Third Point}: A brief description of the third point.
\end{itemize}}
    %\slideImageText{slideImageText}{tests/img-1.png}{For example:
\begin{itemize}
    \item \textbf{First Point}: A brief description of the first point.
    \item \textbf{Second Point}: A brief description of the second point.
    \item \textbf{Third Point}: A brief description of the third point.
\end{itemize}}
    %\slideImage{slideImage}{tests/img-1.png}
    %\slideDoubleImage{slideDoubleImage}{tests/img-1.png}{tests/img-1.png}
    %\slideBigImage{tests/img-1.png}

    %% Explore available templates
    %\slideQuote{Some Other Templates to Explore}
    %\slideQuoteText{The centrality of interactions}{For example:
\begin{itemize}
    \item \textbf{First Point}: A brief description of the first point.
    \item \textbf{Second Point}: A brief description of the second point.
    \item \textbf{Third Point}: A brief description of the third point.
\end{itemize}}
    %\slideTextQuote{The centrality of interactions}{For example:
\begin{itemize}
    \item \textbf{First Point}: A brief description of the first point.
    \item \textbf{Second Point}: A brief description of the second point.
    \item \textbf{Third Point}: A brief description of the third point.
\end{itemize}}
    %\slideQA{Could this really work?}{A: It all depends}
    %\slideLiterature{tests/book-cover.png}{http://somewhere.com/book}{\begin{itemize}
    \item Chapter 1
    \item Chapter 2
    \item Chapter 3
    \item Chapter 4
    \item Chapter 5
    \item Chapter 6
\end{itemize}
}


    \slideConcept[title=slideConcept, font=big, one=72\%]{For example:
\begin{itemize}
    \item \textbf{First Point}: A brief description of the first point.
    \item \textbf{Second Point}: A brief description of the second point.
    \item \textbf{Third Point}: A brief description of the third point.
\end{itemize}}
    \slideConceptTwo[title=slideConceptTwo, font=big, color=true, one=72\%, two=25\%]{For example:
\begin{itemize}
    \item \textbf{First Point}: A brief description of the first point.
    \item \textbf{Second Point}: A brief description of the second point.
    \item \textbf{Third Point}: A brief description of the third point.
\end{itemize}}{For example:
\begin{itemize}
    \item \textbf{First Point}: A brief description of the first point.
    \item \textbf{Second Point}: A brief description of the second point.
    \item \textbf{Third Point}: A brief description of the third point.
\end{itemize}}
    \slideConceptThree[title=slideConceptThree, font=larger, one=72\%, two=25\%, three=19.5\%]{Explanation 1}{Explanation 2}{Explanation 3}
    \slideConceptFour[title=slideConceptFour, font=large, one=72, two=25, three=19.5, four=10]{Explanation 1}{Explanation 2}{Explanation 3}{Explanation 4}
    
    %% Close the Slide Deck
    %\slideBackCover
\end{document}




